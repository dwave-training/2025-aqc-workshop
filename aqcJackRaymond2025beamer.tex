
%\pdfminorversion=4
\pdfminorversion=4
\documentclass[aspectratio=169,final,11pt,forpublic]{beamer} %
% Options:
% [final] Final version.  Removes CONFIDENTIAL DRAFT marking.  Default false.
% [forpublic] Non-confidential?  Uses copyright footer.  Default false.
% [confidential] Confidential?  Uses confidential footer.  Default false.
% [aspectratio=169] or [aspectratio=43]: Widescreen (16:9) or normal size (4:3)?
%      NOTE: Compile twice if you change the aspect ratio, otherwise things won't look right.


\usepackage{listings}
\usetheme{dwave}
\newcommand\cutcontent[1]{}

\title{Hands-on Workshop with D-Wave Quantum Computers}
\date[2025]{12 June, 2025}
\author{Jack Raymond}

\usepackage{braket,tikz,ulem,xmpmulti,siunitx,algorithm,algorithmic}%
\usetikzlibrary{arrows}
\usetikzlibrary{tikzmark,fit,shapes.geometric}

\definecolor{myorange}{rgb}{0.85000,0.32500,0.09800}%
\definecolor{myblue}{rgb}{0.00000,0.44700,0.74100}%
\definecolor{mypurple}{rgb}{0.49400,0.18400,0.55600}%

\newlength{\figurewidth}
\newlength{\figureheight}

%Allows you to change the text width of a single frame.  This is useful for wide slides
%that don't have wide content.  Example:
%\Narrower[3cm]{ Some content, maybe the whole slide (not counting the title) }
%\Narrower{ Default narrowing is 4cm }
\newcommand\Narrower[2][4cm]{%
\makebox[\linewidth][c]{%
  \begin{minipage}{\dimexpr\textwidth-#1\relax}
  \raggedright#2
  \end{minipage}%
  }%
}

%%%%%%%%%%%%%%%%%%%%%%%%%%%%%%%%%%%%%%%%%%%%%%%%%%%%%%%%%%%%%%%%%%%%%%%%%%%%%%%% 
%%%%%%%%%%%%%%%%%%%%%%%%%%%%%%%%%%%%%%%%%%%%%%%%%%%%%%%%%%%%%%%%%%%%%%%%%%%%%%%% 
%%%%%%%%%%%%%%%%%%%%%%%%%%%%%%%%%%%%%%%%%%%%%%%%%%%%%%%%%%%%%%%%%%%%%%%%%%%%%%%% 
\begin{document}
%\maketitle


%\headshotbackground{Jack Raymond}{Technical Lead, Performance Research}
%\begin{frame}\headshotpic[fill=white]{profile.jpeg}%
  
%\end{frame}

%\setblackbackground{0}
\title{Hands-on Workshop \\ with D-Wave \\ Quantum Computers}
\maketitlering

\setwhitebackground{0}

\setwhitebackground{3}
\begin{frame}\frametitle{\bf In this workshop}
  \begin{columns}
    \column{0.5\linewidth}
    The aim of the workshop is to showcase and work hands-on with some ocean-sdk tools useful for physics experiments\\
    \begin{itemize}
    \item Landau-Zener and Kibble-Zurek experiments
    \item Jupyter notebook, with code snippets and links
    \item Question and answer
    \end{itemize}
    Check your email for an invitation to the Leap project, which provides the API token necessary for D-Wave quantum computer access.\\
    Log in to github, or download the code: http://github.com/dwavesystems/AQC2025workshop
    \column{0.5\linewidth}
    \includegraphics[width=0.8\linewidth]{chip.png}
  \end{columns}
\end{frame}

\begin{frame}\frametitle{\bf Quantum annealing}
  \begin{columns}
    \column{0.5\linewidth}
    \vspace{-0.2cm}
    Evolve a multi-spin system for time $t_a$, $$H(s=t/t_a) = -\Gamma(s) \sum_i \sigma^x_i + \mathcal{J}(s) H_p \nonumber $$
  \begin{itemize}
    \item[1] Program a problem Hamiltonian $H_P = \sum_{ij} J_{ij}\sigma^z_i \sigma^z_j + \sum_i h_i \sigma^z_i$
    \item[2] Prepare a ground state at (large $\Gamma(s=0)$ and small $\mathcal{J}(s=0))$
    \item[3] Evolve to large $\mathcal{J}(s=1)$ and small $\Gamma(s=1)$
    \item[4] Measure in the computational basis
  \end{itemize}
  Kadowaki and Nishimori, Phys. Rev. E 58, 5355 (1998)
  
  \column{0.5\linewidth}
  \includegraphics[width=0.8\linewidth]{Figure4.png}
\end{columns}
\end{frame}

\begin{frame}\frametitle{\bf Six recent papers}
  \begin{columns}
    \column{0.55\linewidth}
    \vspace{-0.2cm}
  \begin{itemize}
    \item[1*] Coherent quantum annealing in
      a programmable 2,000 qubit Ising chain, Nature
      Physics (2022)
    \item[2*] Quantum critical dynamics in
      a 5,000-qubit programmable spin glass, Nature
      (2023)
    \item[3] Quantum error mitigation in quantum annealing, npj Quantum Information (2025)
    \item[4] Beyond classical computation in quantum simulation, Science (2025)
    \item[5] Blockchain with proof of quantum work (arXiv:2503.14462)
    \item[6] Quantum dynamics in frustrated Ising fullerenes (arXiv:2505.08994)
  \end{itemize}
  \column{0.45\linewidth}
  \includegraphics[width=\linewidth]{supremacy.png}
\end{columns}
\end{frame}


\begin{frame}\frametitle{\bf Landau-Zener dynamics with 16 qubit random models}
  The probability to excite away from the ground state decays exponentially with the annealing time: the exponent determined by the inverse square gap. 
  \begin{columns}
    \column{0.5\linewidth}
    \begin{center}
    \includegraphics[width = 0.5\linewidth]{adiabatic.png}\\
    \includegraphics[width = 0.8\linewidth]{Figure2A_16q.png}\\
    {\small Quantum critical dynamics in
      a 5,000-qubit programmable spin glass, Nature
      (2023)}
    \end{center}
    \column{0.5\linewidth}
    \includegraphics[width = 0.6\linewidth]{Figure2C_16q.png}
    \end{columns}
\end{frame}

\begin{frame}\frametitle{\bf Kibble-Zurek dynamics in 1D}
  Defect rates (equivalently residual energy) decays as a power of the annealing time; the power determined by the universality class of the phase transition.
  \begin{columns}
    \column{0.5\linewidth}
    \begin{center}
      \includegraphics[width = \linewidth]{Coherent2000_Figure1a.png}\\
      {\small Coherent quantum annealing in
      a programmable 2,000 qubit Ising chain, Nature
      Physics (2022)}
    \end{center}
    \column{0.5\linewidth}
    \begin{center}
      \includegraphics[width = \linewidth]{Coherent2000_Figure1bc.png}\\
    \end{center}
\end{columns}
\end{frame}

%\begin{frame}\frametitle{\bf Implementing experiments (AQC2024)}
%  \begin{columns}
%    \column{0.5\linewidth}
%    \vspace{-0.2cm}
%    \begin{itemize}
%      \item Fast anneal newly (AQC 2024!) released in all online QPU solvers
        
%      \item Allows simple reproduction of earlier results
%
%      \item E.g. Probability of the ground state
%  
%      \item 3 spin-glass models (different colors):
%        16 variables, 52 couplers (+/-1), various gaps
%
%      \item Schedule $\{\Gamma(s), \mathcal{J}(s)\}$ as published on the website; Matching annealing\_time; $J$ modelled == $J$ programmed
%        
%      \item Lines are exact closed system dynamics, points are single-programming QA data
%  \end{itemize}
%  \column{0.5\linewidth}
%  \begin{center}
%  \includegraphics[width=0.8\linewidth]{Figure2C.pdf}\\
%  (reproduction of Figure 2c) \\
%  {\small Quantum critical dynamics in
%  a 5,000-qubit programmable spin glass, Nature
%  (2023)}
%  \end{center}
%\end{columns}
%\end{frame}

%\begin{frame}\frametitle{\bf Under 20 lines of code, a 1D experiment}
%  \begin{center}
%    \includegraphics[width=0.7\linewidth]{chain1024code.png}\\
%    This Kibble-Zurek (and previous slide Landau-Zener) code, are the subject of the code walk through.
%  \end{center}
%\end{frame}


\begin{frame}\frametitle{\bf Run the code (locally or codespaces)}
  At github.com/jackraymond/AQC2025workshop \\
  Code > codespace > +\\
  \begin{itemize}
    \item[] {\bf dwave setup - -oob}  \# Paste the workshop token for the Leap quantum cloud service access
    \item[] {\bf python main.py}  \# Run with defaults
    \item[] {\bf python main.py - - help}  \# Show experiment options:
  \end{itemize}

  \begin{itemize}
  \item solver\_name: e.g. Advantage2\_system1.1
  \item model: 'Landau-Zener' or 'Kibble-Zurek'
  \item use\_srt: Randomize (sign flip) the computational basis definition. A means to mitigate control errors.
  \item parallelize\_embedding: embed many times. A means to mitigate control, and sampling, errors. 
  \end{itemize}
  Further code examples: physics-experiments-ocean-sdk.ipynb
\end{frame}
  
%\begin{frame}\frametitle{\bf Conclusion}
%
%    \begin{center}
%      \includegraphics[width=0.8\linewidth]{Demo.png}
%    \end{center}
%    Use the latest features : Contribute to open source : Experiment\\
%    Thanks for attending!
%\end{frame}

\end{document}
%%% Local Variables:
%%% mode: latex
%%% TeX-master: t
%%% End:
